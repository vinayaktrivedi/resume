%!TEX TS-program = xelatex
%!TEX encoding = UTF-8 Unicode
% Awesome CV LaTeX Template for CV/Resume
%
% This template has been downloaded from:
% https://github.com/posquit0/Awesome-CV
%
% Author:
% Claud D. Park <posquit0.bj@gmail.com>
% http://www.posquit0.com
%
% Template license:
% CC BY-SA 4.0 (https://creativecommons.org/licenses/by-sa/4.0/)
%


\documentclass[10pt, a4paper]{awesome-cv}
\geometry{left=1.4cm, top=.8cm, right=1.4cm, bottom=1.8cm, footskip=.5cm}
\fontdir[fonts/]

% Color for highlights
% Awesome Colors: awesome-emerald, awesome-skyblue, awesome-red, awesome-pink, awesome-orange
%                 awesome-nephritis, awesome-concrete, awesome-darknight
\colorlet{awesome}{awesome-darknight}
% Uncomment if you would like to specify your own color
% \definecolor{awesome}{HTML}{CA63A8}

% Colors for text
% Uncomment if you would like to specify your own color
% \definecolor{darktext}{HTML}{414141}
% \definecolor{text}{HTML}{333333}
% \definecolor{graytext}{HTML}{5D5D5D}
% \definecolor{lighttext}{HTML}{999999}

% Set false if you don't want to highlight section with awesome color
\setbool{acvSectionColorHighlight}{false}

% If you would like to change the social information separator from a pipe (|) to something else
\renewcommand{\acvHeaderSocialSep}{\quad\textbar\quad}


% Available options: circle|rectangle,edge/noedge,left/right
% \photo{./profile.png}
\name{Yash}{Srivastav}
\position{Senior Undergraduate{\enskip\cdotp\enskip}Computer Science and Engineering}
\address{Indian Institute of Technology, Kanpur}
\mobile{(+91) 705-413-3662}
\email{yash111998@gmail.com}
\homepage{yashsriv.org}
\github{yashsriv}
\linkedin{yashsriv}
% \twitter{@therealyashsriv}
% \quote{``There is no fate but what we make."}

\newcommand{\smallcventry}[6]{\cventry{#1}{#2}{#3}{#4}{#6}}
\newcommand{\specialcvsection}[1]{\cvsection{#1}}




\begin{document}
\makecvheader
\makecvfooter
  {}
  {}
  {\thepage}

\specialcvsection{Educational Qualifications}

\newcommand{\education}[4]{
  & #1 & #2 & &#3 & #4
}
\begin{center}
\begin{tabular}{ | L{0.05cm} l | L{3cm} | L{0.05cm} C{7cm} | r |}
  \hline
  \education{\textbf{Year}}{\textbf{Degree}}{\textbf{Institution(Board)}}{\textbf{CGPA/\%}}\\
  \hline
  \education{July'16 -- June'20 (expected)}{B.Tech, CSE}{Indian Institute of Technology, Kanpur}{9.0/10.0}\\
  \education{2016}{XII}{Star International School, Ranchi (CBSE)}{92.0\%}\\
  \education{2014}{X}{JNV East Singhbhum (CBSE)}{10.0/10.0}\\
  \hline
\end{tabular}
\end{center}
\vspace{-4mm}

%%% Local Variables:
%%% mode: latex
%%% TeX-master: "../cv.tex"
%%% TeX-engine: xelatex
%%% End:
\cvsection{Scholastic Achievements}
\begin{cvhonors}

  \cvhonor
  {\ifdefined \ONEPAGE \else All India \fi AIR 147}
  {JEE Advanced 2016}
  {}

  

  \cvhonor
  {Got A* Grade for Exceptional Performance in Courses}
  {\textbf{Operating Systems} and \textbf{Computer Systems Security}}{}

  \cvhonor
  {Certificate for \textbf{Excellent Tutor}, ESC101 (Introduction to Programming), IIT Kanpur}
  {2020}{}

  \cvhonor
  {Academic Excellence Award by IIT Kanpur}
  {2016}{}

  \cvhonor
  {All India Rank 70}
  {KVPY 2016 (SX Stream)}{}

\end{cvhonors}

%%% Local Variables:
%%% mode: latex
%%% TeX-engine: xetex
%%% TeX-master: "../cv"
%%% End:
\cvsection{Work Experience}
\begin{cventries}

  \cventry
  {Sprinklr}
  {Research And Development Intern}
  {India}
  {May 2019 - July 2019}
  {
    \begin{cvitems}
      \item Worked on a generic Profiler in python, to pin-point Java Threads blocking CPU/IO/Network in a system on Spike, by concurrently using Jstack and top utility. Devised a multi-variable scoring system for threads to reduce false-positives, after a series of observations.
      \item	Currently in Production and helped bring down the CPU usage from \textbf{90\% during Spike to 50\%} by pointing out the inefficient code blocks. 
      \item Implemented a Mongo based Cache for metadata of Logs stored in S3, to reduce the Query time of Spark. Reduced the initial Response time of \textbf{120 seconds to 3 seconds}.
    \end{cvitems}
  }

  \cventry
  {PharmEasy}
  {Tech Development Intern}
  {India}
  {May 2018 - July 2018}
  {
    \begin{cvitems}
      \item Implemented a Redis-based Cache, after performing benchmarking of in-memory Databases namely Aerospike, Memcached and Redis, to Cache the common Database queries. Performed Load-testing of the Cache framework and devised an Adaptive strategy for TTL of the Cache based on access patterns of the Table.
      \item Achieved about \textbf{50 percent drop} in Database queries of tables cached, and \textbf{25
percent drop} in response-time.
    \end{cvitems}
  }

\end{cventries}
\vspace{-2mm}

%%% Local Variables:
%%% mode: latex
%%% TeX-engine: xetex
%%% TeX-master: "../cv.tex"
%%% End:
\cvsection{Skills}
\ifdefined\ONEPAGE
\\
\textbf{Proficient}: C, Golang, Python, Javascript\\
\textbf{Experienced}: C++, Java, Scala, Android\\
\textbf{Exposure}: Haskell, Rust, Dart, Perl\\
\textbf{Web}: Angular, Akka, TypeScript, Redux, Flutter\\
\textbf{Utilities}: Shell Utilities, Git, Docker, Ansible, PostgreSQL, MongoDB, OpenCV,
\LaTeX, Vim, Emacs, Vagrant

\else
\begin{cvskills}

  \cvskill
  {Proficient}
  {C, Golang, Python, Javascript}

  \cvskill
  {Experienced}
  {C++, Java, Scala, Android}
  
  \cvskill
  {Exposure}
  {Haskell, Rust, Dart, Perl}
  
  \cvskill
  {Frameworks}
  {Express.js with Node.js, Akka with Scala, JavaScript, TypeScript, Angular,
    Redux, Flutter}

  \cvskill
  {Utilities}
  {Linux shell utilities, Git, Docker, Ansible, Postgres,
    MongoDB, OpenCV, \LaTeX, Vim, Emacs, vagrant}

\end{cvskills}
\fi
%%% Local Variables:
%%% mode: latex
%%% End:
\cvsection{Relevant Courses}

\ifdefined\ONEPAGE

% \textbf{CS:} Introduction to Programming(A$*$), Logic in Computer
% Science, Computer Organization, Data Structures and Algorithms, Computing
% Laboratories - 1(A$*$)
\begin{tabular*}{\textwidth}{l l l l}
  Introduction to Programming(A$*$) & Discrete Mathematics  & Computer
                                                              Organization &
                                                                             Computer Architecture \\
  Data Structures and Algorithms & Probability \& Statistics(A$*$) & Computing
                                                                     Lab
                                                                     - 1(A$*$) &
                                                                                 Computing
                                                                                 Lab
                                                                                 -
                                                                                 2(A$*$) \\
  Compiler Design &  Functional Programming(A$*$) & Computer Systems Security & Computer Networks($i$)
\end{tabular*}



% \textbf{Math}: Discrete Math, Probability and Statistics(A$*$)


{\footnotesize
    {A$*$: Grade for exceptional performance, $i$: In progress}
}

\else
{\fontsize{11pt}{1em}\bodyfontlight\upshape\color{text}
  \begin{tabular*}{\textwidth}{l l l}
    Introduction to Programming(A$*$) & Discrete Mathematics  & Computer Organization \\
    Computer Architecture & Data Structures and Algorithms & Probability \& Statistics(A$*$) \\ 
    Computing Laboratories - 1(A$*$) & Computing Laboratories - 2(A$*$) & Compiler Design \\
    Functional Programming(A$*$) & Computer Systems Security & Computer Networks($i$)
  \end{tabular*}
}
{\fontsize{11pt}{1em}\footerfont\upshape\color{text}
  \begin{tabular*}{\textwidth}{ l l }
    \entrylocationstyle{A$*$: Grade for exceptional performance} & \entrylocationstyle{$i$: In progress}\\
  \end{tabular*}
}
\vspace{-0.5cm}

\fi

%%% Local Variables:
%%% mode: latex
%%% TeX-engine: xetex
%%% TeX-master: "../cv"
%%% End:
\newpage
\cvsection{Projects}

\begin{cventries}

  \cventry
  {Course Project - CS665}
  {\href{https://github.com/mayanksha/CS665-RA}{HackTheL3}}
  {\emph{\texttt{\href{https://github.com/mayanksha/CS665-RA}{github://mayanksha/CS665-RA}}}}
  {September, 2019}
  {
    \begin{cvitems}
    \item A Research Project on Reverse-Engineering the Cache Replacement Policy of L3 Cache, using various cache-set and slice aware data-access patterns.
    \end{cvitems}
  }

  \cventry
  {E-Cell, IITK}
  {SIP Portal}
  {}
  {December 2017}
  {
    \begin{cvitems}
    \item An Automation Portal which acts as a channel amongst Startups, Students and College for facilitating Internships. 
    \item Actively used by SIP, IITK since two years with 500+ Student Registrations each year.
    \end{cvitems}
  }

  \cventry
  {Microsoft Codefundo++}
  {Alertify}
  {\emph{\texttt{\href{https://github.com/vinayaktrivedi/alertify}{github://vinayaktrivedi/alertify}}}}
  {Oct'2018}
  {
    \begin{cvitems}
      \item A Web App on Django, to Alert users about impending disasters in real-time based on their locations, using Text-processing methods like Stemming, Keyword Extraction and Sentiment Analysis on Tweets. 
      \item \textbf{Runner-up in Microsoft Codefundo++ 2018.}
    \end{cvitems}
  }

  \cventry
  {Course Project - CS252}
  {Bhagvada Geeta ChatApp}
  {}
  {October 2018}
  {
    \begin{cvitems}
    \item A Mobile app to display the Geeta-shlokas relevant to your problem query, using various NLP Techniques and 7-dimensional representation of Shlokas on 7-sins. 
    \end{cvitems}
  }

  \smallcventry
  {An Interpreter for Oz}
  {Ozit}
  {written in Oz.}
  {Sept'2019}
  {{\emph{\texttt{\href{https://github.com/vinayaktrivedi/ozit}{github://Ozit}}}}}
  {}

  \smallcventry
  {Compiler for Golang}
  {Gorakshak}
  {written in Python.}
  {Jan'2019}
  {{\emph{\texttt{\href{https://github.com/vinayaktrivedi/GOrakshak}{github://GOrakshak}}}}}
  {}

  \smallcventry
  {Course Project}
  {GemOS}
  {Operating Systems}
  {Aug'2018}
  {{\emph{\texttt{\href{https://github.com/vinayaktrivedi/gemOS}{github://https://gemOS}}}}}
  {}

  \smallcventry
  {NodeJs+PHP backend based}
  {VAAD SAMVAAD}
  {Social Networking website.}
  {Jan'2018}
  {{\emph{\texttt{\href{https://github.com/vinayaktrivedi/facebook-copy}{github://facebook\_copy}}}}}
  {}

  \smallcventry
  {A cryptographically Secure Cloud Storage Web App}
  {Cloud2.0}
  {written in Golang.}
  {}
  {{\emph{\texttt{\href{https://github.com/vinayaktrivedi/Cloud2.0}{github://Cloud2.0}}}}}
  {}

\end{cventries}
\vspace{-1mm}

%%% Local Variables:
%%% mode: latex
%%% TeX-master: "../cv.tex"
%%% TeX-engine: xelatex
%%% End:
\cvsection{Positions of Responsibility}

\begin{itemize}
\item \textbf{Head, Web}, \emph{Students' Placement Office, IITK} (2019-2020)
	System Admin of SPO Servers, Responsible for Scaling, Deployment And Maintenance of various Automation Portals. Introduced many Automation features like Scraping of JAFs and core functionalities like \textbf{Walk-In Interview, Mail-on-news, Tracker} etc. 

\item \textbf{Tutor, ESC101} \emph{(Introduction to Programming)} \textbf{(Twice)}

   Responsible for conducting Tutorial sessions for a batch of 40 students, and helping to set up Lab Assignments, Exams and Quizzes for a class of Strength \textasciitilde 500.
  

\end{itemize}
\vspace{-1mm}

\cvsection{Miscellaneous}

\begin{itemize}
  \item Developed AM Portal, A PHP backend based Q/A Forum Portal for college.
  \item Mentored 6 students to build a custom package manager. 
  \item Mentored 2 students to build a light nginx-like server.
  \item Mentored 12 students to build a Movie Recommender App.
  %\item Developed MVP for an entrepreneurial idea ICO. 
    \vspace{-1mm}
\end{itemize}
% \cvsection{Interests}

{\fontsize{11pt}{1em}\bodyfontlight\upshape\color{text}
  \begin{itemize}
  \item Open Source
  \item Capture The Flag Contests
  \item Web Development
  \item Image Processing
  \item Artificial Intelligence
  \item Robotics
  \end{itemize}
}

%%% Local Variables:
%%% mode: latex
%%% TeX-engine: xetex
%%% TeX-master: "../cv"
%%% End:


\end{document}

%%% Local Variables:
%%% mode: latex
%%% TeX-engine: xetex
%%% End: